\documentclass[uplatex,b5j]{jsarticle} % uplatexオプションを入れる
%\documentclass[b5j]{ujarticle}        % jarticle系にしたい場合はこっち
\usepackage[scale=.8]{geometry}      % ページレイアウトを変更してみる
\usepackage[T1]{fontenc}             % T1エンコーディングにしてみる
\usepackage{txfonts}                 % 欧文フォントを変えてみる
\usepackage{plext}                   % pLaTeX付属の縦書き支援パッケージ
\usepackage{okumacro}                % jsclassesに同梱のパッケージ
\begin{document}
\title{とにかくup{\LaTeX}を使ってみる}
\author{コピペ太郎}
\西暦\maketitle                      % 漢字のマクロ名もOK

\section{日本語と数式}
\textbf{ゼータ関数}(zeta function)というのは
\begin{equation}                     % 数式中の漢字もOK
\zeta(s) \stackrel{定義}{=} \sum_{n=1}^\infty \frac{1}{n^s}
= \prod_{p\colon 素数}\frac{1}{1-p^{-s}}
\end{equation}
とかいう奴のこと.

\section{縦書き}
\setlength{\fboxsep}{.5zw}
縦書きの例は
\fbox{\parbox<t>{12zw}{%
  \setlength{\parindent}{1zw}
  {\TeX}(テック,テフ)はStanford大学のKnuth教授に
  よって開発された組版システムである.
  {p\TeX}は\株 アスキーが{\TeX}を日本語対応
  (縦書きを含む)にしたものである.
  \par\bigskip
  このように\bou{縦書き}についても{p\TeX}と
  全く同様に文書を作成できます.
  \par\bigskip
  平成\rensuji{20}年\rensuji{4}月\rensuji{1}日
  \par\medskip
  平成\kanji20年\kanji4月\kanji1日
}}%
こんなの.

\section{okumacroで遊んでみる}
\ruby{組}{くみ}\ruby{版}{はん},\ruby{等}{とう}\ruby{幅}{はば},
\keytop{Ctrl}+\keytop{A} \keytop{Del}
\keytop{Ctrl}+\keytop{S} \return,
\MARU{1}\MARU{2}\MARU{3}
\par\bigskip
\begin{shadebox}
\挨拶 それでは.敬具
\end{shadebox}

\end{document}
